%*****************************************************************
% Breve riassunto in italiano della tesi da cui si capisca tutto
% ****************************************************************
\newcommand{\estrattoname}{Estratto}
\addcontentsline{toc}{chapter}{\estrattoname}

\pdfbookmark[1]{Estratto}{Estratto}
\begingroup
\let\clearpage\relax
\let\cleardoublepage\relax
\let\cleardoublepage\relax

\chapter*{Estratto}
Lo scopo di questa tesi è quello di studiare l'architettura di un driver per motori brushless commerciale proposto da Texas Instruments e implementato come un circuito elettronico come un progetto "open source" disponibile online, analizzando i vantaggi e gli svantaggi di tale progetto per quanto riguarda l'implementazione di eccitazione a motore trapezoidale e sinusoidale e tecniche di controllo di velocità e corrente per un veicolo senza pilota progettato per l'agricoltura robotica. Successivamente, l'implementazione di tali tecniche di eccitazione e controllo è stata effettuata e testata fisicamente.

Questa tesi spiega in dettaglio le parti più importanti riguardanti l'implementazione fisica di un sistema di guida del motore in modo tale da poter essere completamente replicato. Il capitolo \ref{chap:intro} contiene alcune informazioni iniziali sulla storia del motore elettrico e la motivazione per realizzare questo lavoro. Nel capitolo \ref{chap:problem}, spieghiamo più in dettaglio i diversi motivi per cui questo lavoro è stato sviluppato e gli aspetti principali che sono stati evidenziati. Nel capitolo \ref{chap:theory} viene fornita una spiegazione semplice ma sufficiente sulla teoria alla base del motore elettrico, che spiega anche le diverse tecnologie esistenti e i loro particolari metodi di guida. Nel capitolo \ref{chap:robi} spieghiamo il caso di studio di ROBI', un prototipo di manipolatore mobile per applicazioni agricole, che utilizza il motore a ruote motrici per il quale è stato sviluppato il driver motorio di questo lavoro. Nel capitolo \ref{chap:implementation} spieghiamo dettagliatamente tutto il lavoro sviluppato attorno all'implementazione del driver del motore, sia nel campo del software che in quello dell'hardware. Nel capitolo \ref{chap:results} spieghiamo i risultati finali del lavoro, mostrando e confrontando grafici e forme d'onda del comportamento del motore a ruote motrici guidato dal sistema sviluppato. Infine, nel capitolo \ref{chap:conclusion}, discutiamo i risultati del lavoro realizzato e proponiamo un nuovo sistema, basato sull'implementazione dell'architettura ricercata e sui problemi incontrati durante il lavoro sul progetto.


\endgroup

