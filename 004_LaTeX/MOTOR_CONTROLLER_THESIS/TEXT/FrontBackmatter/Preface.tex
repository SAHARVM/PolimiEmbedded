
\addcontentsline{toc}{chapter}{\prefacename}
\pdfbookmark[1]{Preface}{Preface}

\chapter*{Preface}
\begin{flushright}{\slshape    
   Knowledge is only part of understanding.\\
   Genuine understanding comes from hands-on experience}.
   \\ \medskip --- \citeauthor{constructionism}
   \citetitle{constructionism} \citeyear{constructionism}
\end{flushright} 

Motor control is a topic that must be experienced personally to be understood. This is a characteristic of many other engineering topics: they need to be experienced by the engineer or by the student to be fully understood. All the theory behind the movement of the shaft of the electric motor, which explains all the different phenomena interacting to create mechanical motion from electrical energy, should be experienced to fully understand everything that is involved. This is the reason why the practical implementation of theoretical topics is always interesting. Practical implementation makes us realize that there are always challenges that might not be taken into account while they are being studied from books, and they represent new oportunities to drive research and development for improvement.

This text was written with the idea of becoming a guide on the development of a motor controller for further projects, including the explanation of the basic physical phenomena that acts on electric motors and the important parameters to consider for the prediction of its motion, to the implementation of the hardware and software of a driving circuit and a detailed explanation on the most important factors to bring a motor controller alive and to successfully drive a permanent magnet motor. Even if the development of the work was intended for a specific project, all the information related to the development of the motor control systems can be applied to different projects, which is why this work can be useful also as a reference for projects not related to the robotic agriculture.\\

--- Arturo Mont\'ufar Arreola